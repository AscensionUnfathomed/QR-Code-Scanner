%!TEX root = bericht.tex
%!TEX TS-options = -shell-escape
%!BIB programm = biber
%----------------------------------------------------------------------------------------
%-- diese Datei ist für kurze eigene Befehle gedacht, um zum Beispiel Namen von geschützten 
%-- Eigenmarken oder Algorithmen konsistenz zu setzen. Zu diesem Zwecke wird diese Datei
%-- von git auch immer mittels "union" gemerged! Außerdem ist dies ein guter Platz für 
%-- eigene Mathebefehle.
%----------------------------------------------------------------------------------------

\DeclareRobustCommand{\Simrank}{\textsc{SimRank}\xspace}
\DeclareRobustCommand{\SimrankPP}{\textsc{SimRank}\nolinebreak[4]\hspace{-.09em}+\hspace{-.15em}+\xspace}
\DeclareRobustCommand{\Cosimrank}{\textsc{CoSimRank}\xspace}
\DeclareRobustCommand{\Pagerank}{\textsc{PageRank}\xspace}

\DeclareRobustCommand{\LokiSports}{\enquote{lokisports}\xspace}
\DeclareRobustCommand{\FirstSpirit}{\textsc{FirstSpirit}\xspace}
\DeclareRobustCommand{\Woopra}{\textsc{Woopra}\xspace}

\DeclareRobustCommand{\git}{\textsf{git}\xspace}
\DeclareRobustCommand{\Maven}{\textsf{Maven}\xspace}
\DeclareRobustCommand{\GitLab}{\textsf{GitLab}\xspace}
\DeclareRobustCommand{\Postgres}{\textsf{PostgreSQL}\xspace}
\DeclareRobustCommand{\jOOQ}{\textsf{jOOQ}\xspace}
\DeclareRobustCommand{\JDBC}{\textsf{JDBC}\xspace}
\DeclareRobustCommand{\JUnit}{\textsf{JUnit}\xspace}
\DeclareRobustCommand{\Ant}{\textsf{Ant}\xspace}


%-- Mathebefehle
\DeclarePairedDelimiter{\abs}{|}{|}
\DeclarePairedDelimiter{\enbrace}{(}{)}

%-- Um GitLab-Issues einfach zu referenzieren
\newcommand{\issue}[1]{\href{https://algo-git.uni-muenster.de/projektseminar-cms/bericht/issues/#1}{\texttt{\faGitlab~\##1}}}

