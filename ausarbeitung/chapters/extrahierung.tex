\chapter{Extrahierung}

Der nächste Schritt auf dem Weg zum finden eines gültigen \QRCodes besteht darin alle möglichen Dreier Kombinationen von \fps zu prüfen und festzustellen ob diese zur Extrahierung eines gültigen \QRCodes führen. Dies führt zu $\binom{n}{3}$ möglichen Kombinationen von \fps, wobei $n$ die Anzahl aller gefundenen \fps ist. Für Bilder mit großem \emph{n} kann dies schnell zu großem Rechenaufwand führen. Deswegen gibt es im folgenden mehrere Stellen an denen Kombinationen verworfen werden, sobald deutlich wird das diese zu keinem korrekten Ergebnis führen können.

\section{Pattern Positionierung} \todo{Besserer Section Titel.}
Hat man drei \fps ausgewählt, muss für die korrekte Extrahierung festgestellt werden welche Position im \QRCode jedes \fp einnimmt. Dafür ordnen wir sie im Uhrzeigersinn an unter Verwendung der Gaußschen Trapezformel
$$ A=\frac{1}{2} \sum_{0}^{n-1} (y_i + y_{i+1\ mod\ n})(x_i - x_{i+1\ mod\ n}) $$
Diese Formel berechnet den Flächeninhalt eines einfachen Polygons. Dabei ist $n$ die Anzahl der Punkte in unserem Polygon und $x_i$ und $y_i$ die Punktkoordinaten. Abhängig vom Drehsinn der Polygon Punkte im Bezug auf das Koordinatensystem ist dieser Flächeninhalt entweder positiv oder negativ. In einem geodätischen Koordinatensystem entspricht ein positiver Flächeninhalt dabei einem Drehsinn im Uhrzeigersinn und ein negativer gegen den Uhrzeigersinn.

Angewendet wird die Formel auf das Polygon, welches aus dem jeweils ersten Punkt der Kontur jedes \fps besteht. Sind die \fps in korrekter Reihenfolge, bestimmen wir als nächstes das obere linke \fp.

Zurzeit haben wir durch die drei Patterns insgesamt 12 approximierte Kanten. Da die Finder Patterns in einem korrekten QRCode allerdings so ausgerichtet sind, dass das obere linke Pattern ausschließlich Kanten hat die mit einem der anderen beiden Patterns identisch sind, haben wir nur 8 einzigartige Kanten. Stellt man also fest welche der 12 Geraden die gleiche Kantengerade im QRCode beschreiben kann festgestellt werden, welches Pattern das obere linke Pattern ist.

Um dies effizient zu tun, nutzen wir aus das die Stützvektoren der Kantengeraden stets im  Mittelpunkt der Segmente und somit auch ungefähr im Mittelpunkt der echten Kanten des Patterns liegen. Ist dies nämlich der Fall kann man ein Ähnlichkeitsmaß für zwei Geraden definieren über die Summe der Entfernungen der Stützvektoren einer Gerade zur anderen Gerade. (Bild für Verständnis) Ein kleinerer Wert bedeutet in diesem Fall dann das sich zwei Geraden besonders ähnlich sind. In Abbildung (Bla) ist dabei gut zu erkennen, dass wenn es sich um einen QRCode handelt, welcher mit ausreichend guter Qualität für eine korrekte Kanten approximation abgebildet ist, das sich dieses Maß sehr gut eignet um festzustellen welche der insgesamt 12 Kantengeraden die gleiche Kante approximieren. Für das obere linke Pattern gilt dann in diesem Fall, dass es das Pattern mit den insgesamt vier kleinsten Ähnlichkeitswerten ist.

Da zuvor die Patterns im Uhrzeigersinn angeordnet wurden, können wir durch das Rotieren im Uhrzeigersinn nicht nur das obere linke Pattern an die erste Stelle des Arrays bringen, sondern wissen auch direkt das sich an der zweiten Stelle das obere rechte Pattern und an der dritten Stelle das untere linke Pattern befindet. Mit der Ausnahme natürlich für Spiegel verkehrte QRCodes von denen es hier nicht das Ziel ist diese auch zu erkennen.

Da wir bereits berechnet haben, welche der Kantengerade die selbe Kante approximieren, können wir diese Informationen benutzen um eine qualitativ deutlich bessere Approximation für alle Kanten des oberen linken Patterns zu berechnen. Wir nehmen dazu die zugrunde liegenden Segmente von den gepaarten Kanten und führen auf der Menge dieser Punkte erneut eine Approximation der Kantengeraden durch. Indem wir dabei feststellen ob wir mit einer Kante aus dem oberen rechten oder dem unteren linken Pattern vereinigen, können wir feststellen ob die neue Kante im QRCode Koordinatensystem eine horizontale oder vertikale Kante beschreibt. Die nicht verwendeten Kanten am Ende des Prozesses sind danach ebenfalls trivial zuzuordnen. (Falls an dieser Stelle nicht 4-2-2 gemerged wurde, abbrechen.)

An dieser Stelle kennen wir dann vier horizontale und vier vertikale Kantengeraden des QRCodes die durch die Finder Patterns beschrieben werden.

(Sagen dass wir das machen weil die anderen Mehtoden nicht zu 100% perspektivisch sicher, bzw. Berechnungstechnisch weniger aufwändig sind.) Um festzustellen welche von diesen Kanten die Äußeren Kanten sind, sortieren wir jeweiles vier Kanten einer Richtung entlang einer beliebigen Kante der anderen Richtung. Dazu berechnen wir die Schnittpunkte der vier Geraden mit der gewählten Achse, rotieren diese, sodass sich alle Schnittpunkte auf der X-Achse bewegen (also rotation ins Achsensystem) und sortieren dann schlicht nach kleinstem zu größten x wert.

Nachdem dies